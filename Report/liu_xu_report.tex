\documentclass[11pt]{article}
\usepackage{indentfirst}
\usepackage{amsmath}

\title{SWEN90004 Modelling Complex Software System\\
        \textbf{Assignment 2 Report}}
\author{Zheping Liu, 683781, zhepingl\\
        Zewen Xu, 862393, zewenx}
\date{}

%TODO: This is a comment in Latex

\begin{document}
    \maketitle
    \section{Background}
        The chosen model for our analysis is the \textbf{Rebellion} model. 
        This model is based on model of civil violence by Joshua Epstein (2002).
        The aim of this project is to replicate this model and study how this 
        complex system behaves in different stages and under different settings.
        It describes the how agents behave against central authority in relation
        to their grievance and the power of authority.
        \paragraph{}
        \textbf{Rebellion} is a complex system. Firstly, this model is made
        of many individual instances, each of them has individual state, but at
        the same time, their behaviour (i.e. state transitions) are affected by
        other states of other instances near them. Secondly, the states of the 
        whole system is unpredictable at a certain time t with a given set of inputs.
        This is due to the interrelation and interactions between states of 
        different instances and certain degree of randomness in this model. 
        Thirdly, under most settings, the model is decentralised. The vision of
        most instances cannot cover the whole board. Also, all agents are the same, 
        and all cops are the same, none of them is a leader in this model. This
        implies they all contribute the same amount to influence the states of the
        model. Fourthly, there is feedback on the behaviour of agents. Active agents
        can eventually influence agents around them and lead to more active agents.
        Also, more active agents will attract more cops to enforce on them, which
        leads to more quiet agents. Finally, although there exists regular patterns 
        for states of the system, but it does not approach to an equilibrium. 
        
    \section{Model}
        The model includes the following major components:
        \subsection{Board}
        Board is the representation of the world in this model. It is set in default
        to contain $1600\:(40 \times 40)$ patches in total.
        \subsubsection{Patches}
        Each patch in the board has two states, \textit{empty} and \textit{occupied}.
        \subsection{Agent}
        Agents are the representation of ordinary individuals in this model. Agents
        have three states, \textit{active}, \textit{quiet} and \textit{jailed}. 
        They are default to be \textit{quiet} at the beginning. Agents will be able
        to move to any empty patches in their vision when \textit{MOVEMENT} value
        equals to \textit{true} every tick. They will update their state after the
        movement phase according to the following equation.
        \begin{equation}
            \begin{split}
                grievance\:-\:riskAversion\:\times\:estimatedArrestProbability\:>\:threshold \\
                \text{Where}\ grievance\:=\:perceivedHardship\:\times\:(1\:-\:governmentLegitimacy)\\
                \text{and}\ estimatedArrestProbability = 1 - e^{-k \times ((1 + activeCount) / copsCount)}
            \end{split}
        \end{equation}  
        When this equation holds, their state will become \textit{active}, otherwise
        it will be \textit{quiet}. Both the number of active agents and number of
        cops around them will be factors that decide their behaviours.
        If active agents are in the vision of cops, there
        will be possibility for them to become \textit{jailed} by the enforce action
        of cops. A random jail term between 0 and the maximum jail term allowed will
        be given to them and they are not able to move or action during that time.
        However, a patch contains jailed agents are still considered as \textit{empty}
        when no other agents or cops on it.
        \subsection{Cop}
        Cops are the representation of power of authority in this model. Cops
        are always allowed to move to empty patches within their vision at 
        every tick. They can enforce at most one active agent within their vision
        at each tick and move to their patches.

    \section{Extension}
        The extension we have implemented is that the government legitimacy will
        increase as the proportion of jailed agents to number of total agents increases.
        The equation we derive is 
        \begin{flalign}
          \begin{split}
          GovernmentLegitimacy = \\
          InitialGovernmentLegitimacy + \frac{JailedAgents}{TotalAgents}
          \times (1 - InitialGovernmentLegitimacy)
          \end{split}
        \end{flalign}
    \section{Replication}
    \begin{table}[h]
        \begin{center}
          \begin{tabular}{|l|l|}
          \hline
            Initial cop density & 0.04 \\
          \hline\hline
            Initial agent density & 0.70 \\
          \hline
            Vision & 7 \\
          \hline
            Government Legitimacy & 0.82 \\
          \hline
            Max jail term & 30 \\
          \hline
            Movement & True \\
          \hline
          \end{tabular}
          \caption{Setting No.1}\label{table1}
        \end{center}
      \end{table}
        
    \section{Extension}
    \section{Results \& Discussion}
    \section{Conclusion}
\end{document}