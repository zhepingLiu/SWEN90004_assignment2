\documentclass[11pt]{article}
\usepackage{indentfirst}
\usepackage{amsmath}

\title{SWEN90004 Modelling Complex Software System\\
        \textbf{Assignment 2 Report}}
\author{Zheping Liu, 683781, zhepingl\\
        Zewen Xu, 862393, zewenx}
\date{}

\begin{document}
    \maketitle
    \section{Background}
        The chosen model for our analysis is the \textbf{Rebellion} model. 
        This model is based on model of civil violence by Joshua Epstein (2002).
        The aim of this project is to replicate this model and study how this 
        complex system behaves in different stages and under different settings.
        It describes the how agents behave against central authority in relation
        to their grievance and the power of authority. 
        
    \section{Model}
        The model includes the following major components:
        \subsection{Board}
        Board is the representation of the world in this model. It is set in default
        to contain $1600\:(40 \times 40)$ patches in total.
        \subsubsection{Patches}
        Each patch in the board has two states, \textit{empty} and \textit{occupied}.
        \subsection{Agent}
        Agents are the representation of ordinary individuals in this model. Agents
        have three states, \textit{active}, \textit{quiet} and \textit{jailed}. 
        They are default to be \textit{quiet} at the beginning. Agents will be able
        to move to any empty patches in their vision when \textit{MOVEMENT} value
        equals to \textit{true} every tick. They will update their state after the
        movement phase according to the following equation.
        \begin{equation}
            \begin{split}
                grievance\:-\:riskAversion\:\times\:estimatedArrestProbability\:>\:threshold \\
                \text{Where}\ grievance\:=\:perceivedHardship\:\times\:(1\:-\:governmentLegitimacy)\\
                \text{and}\ estimatedArrestProbability = 
                %TODO: Add equation for the estimatedArrestProbability
            \end{split}
        \end{equation}  
        When this equation holds, their state will become \textit{active}, otherwise
        it will be \textit{quiet}. If active agents are in the vision of cops, there
        will be possibility for them to become \textit{jailed} by the enforce action
        of cops. A random jail term between 0 and the maximum jail term allowed will
        be given to them and they are not able to move or action during that time.
        However, a patch contains jailed agents are still considered as \textit{empty}
        when no other agents or cops on it.
        \subsection{Cop}
        Cops are the representation of power of authority.
        
    \section{Replication}
    \section{Extension}
    \section{Results \& Discussion}
    \section{Conclusion}
\end{document}