\documentclass[UTF8,11pt]{article}
\usepackage{indentfirst}
\usepackage{amsmath}

\title{SWEN90004 Modelling Complex Software System\\
      \textbf{Assignment 2 Proposal}}
\author{Zheping Liu, zhepingl, 683781\\
        Zewen Xu, zewenx, 862393}
\date{}

%1. a description of the model you are replicating:
%    - what is it a model of?
%    - what is it a model for? (ie, what can a user learn from it?)

%2. the *conceptual* design of the existing model:
%    - what are the parts of the system?
%    - what states can they take?
%    - how do the parts interact?
%    - how are states updated?
%    - what are the parameters of the system?

%3. a plan for how you intend to implement this in Java?
%    - what objects, classes, data structures will you use?

%4. (perhaps, if you have already spent some time playing with the NetLogo model) the experiments you intend to run:
%    - what are the key behaviours/dynamics of the system?
%    - how do these behaviours/dynamics change with parameter values?

\begin{document}
    \maketitle
    \section{Descriptive Overview}
        Our group decides to replicate the \textbf{Rebellion} model.
        This model is based on model of civil violence by Joshua Epstein (2002).
        It describes the how agents behave against central authority in relation
        to their grievance and power of authority. Where grievance of agents
        depends on government legitimacy, perceived hardship and risk aversion 
        of each individual. Power of authority is determined by number of police
        officers, their vision and maximum jail terms. There are two global 
        variables in the model that cannot be directly set in the interface that 
        are \textit{k}, the factor for determining arrest probability, and 
        \textit{threshold} for agents to rebel.

        \paragraph{}
        Users can observe how ordinary people feel about the authority under
        certain imposed pressure. In addition, social scientists can study how
        different personalities affect their behaviour to authority.

    \section{Design of Rebellion Model}
        There are two categories of members in this model, \textbf{agents} and 
        \textbf{police officers}. 
        \paragraph{}
        The states of them are made of two parts.
        The first part is their location. Police officers can always move to 
        unoccupied patches freely. Where the mobility of agents can be switched
        on and off in the model.
        \paragraph{}
        The second part is the behaviour of members.
        Police officers only has one state that is arresting active agents 
        (rebelled agents) in their vision. If there are more than one active 
        agents in the vision, they will arrest randomly one of them. There are 
        three different states for agents, quiet, active and jailed. 
        The state update rule are as following:
        \subsection{$Quiet \Rightarrow Active$}
        \begin{equation}
            \begin{split}
                grievance\:-\:riskAversion\:\times\:estimatedArrestProbability\:>\:threshold \\
                \text{Where}\ grievance\:=\:perceivedHardship\:\times\:(1\:-\:governmentLegitimacy)
            \end{split}
        \end{equation}
        \subsection{$Active \Rightarrow Quiet$}
        \begin{equation}
            \begin{split}
                grievance\:-\:riskAversion\:\times\:estimatedArrestProbability\:\leq\:threshold
            \end{split}
        \end{equation}
        \subsection{$Active \Rightarrow Jailed$}
            If police officers find active agents in their vision, they will
            randomly arrest one of these active agents and change their state 
            from \textbf{active} to \textbf{jailed}.
        \subsection{$Jailed \Rightarrow Quiet$}
            If the jailed agents stay in jail after the \textit{maximum jailed 
            period}, they will become quiet.

    \section{Plan of Implementation}
        There are two major components in our architecture. Controller and Model.
        In the model we have Agent class, Patch class and PoliceMan class. In the
        Controller component, there are Board class and Controller class. Board
        is representing all patches in the board, in other words, the map for this
        model. Controller will control the states of all agents and police man.

\end{document}